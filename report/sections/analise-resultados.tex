\section{Análise dos Resultados}

	De acordo com os resultados encontrados, podemos observar que existe um ponto ideal até onde o paralelismo é eficiente. Para maximizar a eficiência, nenhum processo deve ficar ocioso. Portanto devem existir escravos o suficiente para que o tempo de delegação de tarefas do mestre seja exatamente o mesmo tempo que um dos escravos leva para executar sua tarefa, fazendo com que, ao terminar a tarefa, o mestre já esteja preparado para delegar uma nova tarefa ao escravo.
	
	Com muito poucos processos, o mestre, após delegar as tarefas, fica ocioso esperando os escravos terminar, o que não é ótimo em questão de eficiência. Também é importante notar, que depois de um determinado número de escravos, o mestre não é capaz de lidar com todos, fazendo com que alguns escravos fiquem ociosos esperando o mestre, o que também afeta a eficiência do sistema.